\documentclass[12pt,a4paper]{article}
\usepackage[utf8]{inputenc}
\usepackage[english]{babel}
\usepackage{amsmath}
\usepackage{amsfonts}
\usepackage{amssymb}
\usepackage{graphicx}
\author{Hélder Mendes}
\title{Turbofan Engine Models}
\begin{document}

\section{Turbofan Engine Model }



\begin{figure}[ht]
\begin{center}
\includegraphics[width=10cm,scale=1]{figures/engine_model.png}
\caption{xxxxxx}
%\label{vrtg}
\end{center}
\end{figure}

\textbf{Modelo:} \\


%\begin{center}
\begin{equation}
X_{kp} = A \cdot X_{k-1} + B \cdot \underline{u_{k}} + \underline{w_{k}}
\label{new_prediction}
\end{equation}
%\end{center}

\vspace{2cm}

\textbf{Perguntas:} 

\begin{itemize}
\item usar o state vector igual aos measurements? ( X = Y ?) 
\item Os parâmetros de motor que não estão a ser gravados (em falta), não podem fazer parte dos measurements, mas podem ser considerados nas variáveis de estado? Neste caso seriam variáveis escondidas...
\end{itemize}


\end{document}